\documentclass{article}
\usepackage{amsmath}
\usepackage{listings}
\usepackage{xcolor}
\usepackage[a4paper, margin=1in]{geometry}

\definecolor{codebg}{rgb}{0.95,0.95,0.95}
\lstset{backgroundcolor=\color{codebg}, 
        frame=single, 
        basicstyle=\ttfamily,
        keywordstyle=\color{blue},
        commentstyle=\color{gray},
        stringstyle=\color{red},
        showstringspaces=false}

\title{Principal Component Analysis (PCA) and Factor Analysis in R: Comprehensive Guide}
\author{Dani}
\date{}

\begin{document}

\maketitle

\section{Overview of PCA and Factor Analysis}
\textbf{Principal Component Analysis (PCA)} and \textbf{Factor Analysis (FA)} are both multivariate techniques aimed at dimensionality reduction, but with different goals:
\begin{itemize}
    \item \textbf{PCA} focuses on maximizing the variance captured by each component, making it ideal for visualizing high-dimensional data and identifying major sources of variation.
    \item \textbf{FA} seeks to identify latent factors that explain observed variable correlations, making it more suitable for uncovering hidden structures or constructs within the data.
\end{itemize}

\subsection{Choosing Between PCA and FA}
\begin{itemize}
    \item \textbf{Use PCA when}:
        \begin{itemize}
            \item You aim to capture as much variance as possible.
            \item Data is numeric and relationships are linear.
            \item You are more interested in reducing dimensionality for visualization purposes.
        \end{itemize}
    \item \textbf{Use FA when}:
        \begin{itemize}
            \item The goal is to identify underlying latent variables.
            \item You want to explore hidden structures or groupings within the data.
            \item Data might be non-normal or categorical, requiring alternative approaches like polychoric FA.
        \end{itemize}
\end{itemize}

\section{PCA on USArrests Data using \texttt{princomp}}
\begin{itemize}
    \item \texttt{princomp()} uses spectral decomposition on the covariance or correlation matrix.
    \item The choice between using a correlation matrix (\texttt{cor = TRUE}) or covariance matrix depends on whether the data needs to be standardized.
    \item Example:
\begin{lstlisting}[language=R]
data <- USArrests
pcUSA <- princomp(data, cor = TRUE)
summary(pcUSA)

# Eigenvalues of the principal components
eigs <- pcUSA$sdev^2
plot(eigs, type = "b", main = "Scree Plot")

# Loading matrix and biplot
pcUSA$loadings
biplot(pcUSA)
\end{lstlisting}

\item \textbf{Interpretation}:
    \begin{itemize}
        \item The \textbf{Scree plot} helps visualize the proportion of variance explained by each component.
        \item \texttt{pcUSA\$loadings} provides the contribution of each variable to the components.
        \item The first few components should ideally capture the majority of the variance.
    \end{itemize}
\end{itemize}

\subsection{PCA using \texttt{prcomp}}
\begin{itemize}
    \item \texttt{prcomp()} is preferred over \texttt{princomp()} when the dataset is large, as it uses Singular Value Decomposition (SVD).
    \item It provides a numerically stable solution, especially when dealing with highly correlated variables.
    \item Example:
\begin{lstlisting}[language=R]
prusa <- prcomp(data, scale. = TRUE)
summary(prusa)

# Eigenvalues
prusa$sdev^2

# Biplot to visualize the scores and loadings
biplot(prusa)
\end{lstlisting}
\item \textbf{Key Consideration}:
    \begin{itemize}
        \item Use `prcomp()` when numerical stability is a concern, especially for high-dimensional datasets.
    \end{itemize}
\end{itemize}

\section{Factor Analysis using \texttt{FactoMineR}}
\begin{itemize}
    \item \texttt{FactoMineR} is a versatile package for advanced factor analysis and PCA.
    \item It provides additional outputs like communalities, cosines, and variable contributions.
    \item Example on \texttt{USArrests}:
\begin{lstlisting}[language=R]
library(FactoMineR)
usafa <- PCA(USArrests)
summary(usafa)

# Accessing eigenvalues and variable loadings
usafa$eig
usafa$var$coord
\end{lstlisting}

\item \textbf{Practical Tip}:
    \begin{itemize}
        \item Use \texttt{FactoMineR} when you need to assess multiple dimensions, visualize variable contributions, or perform hierarchical clustering on principal components.
    \end{itemize}
\end{itemize}

\section{Bartlett's Test and KMO Index using \texttt{psych}}
\begin{itemize}
    \item \textbf{Bartlett’s Test of Sphericity} tests if the correlation matrix is an identity matrix, indicating if the variables are sufficiently correlated to perform factor analysis.
    \item Example using the Food Price dataset:
\begin{lstlisting}[language=R]
library(psych)
R <- cor(food)
cortest.bartlett(R, n = nrow(food))
\end{lstlisting}

\item \textbf{Kaiser-Meyer-Olkin (KMO) Index} assesses sampling adequacy.
    \begin{itemize}
        \item Values range from 0 to 1, with values above 0.6 indicating that the data is factorable.
    \end{itemize}
\begin{lstlisting}[language=R]
kmo(food)
\end{lstlisting}
\item \textbf{Interpretation of KMO}:
    \begin{itemize}
        \item KMO $>$ 0.9: Excellent
        \item 0.7 $>$ KMO $>$ 0.9: Good to average
        \item KMO $<$ 0.5: Unacceptable
    \end{itemize}
\end{itemize}

\section{Advanced PCA on Food Price Data}
\begin{itemize}
    \item \textbf{Test of Factorability} is essential before PCA to ensure that the variables are correlated.
    \item PCA is performed using the correlation matrix when variables are on different scales.
\begin{lstlisting}[language=R]
pca_food <- princomp(food, cor = TRUE)
summary(pca_food)

# Loadings and visualization
plot(pca_food, type = "lines", main = "Scree Plot for Food Prices")
biplot(pca_food)
\end{lstlisting}

\item \textbf{Interpretation}:
    \begin{itemize}
        \item If the first two components explain a significant proportion of variance (e.g., >70\%), use them for further analysis.
        \item Biplots provide insight into how variables contribute to each component.
    \end{itemize}
\end{itemize}

\section{Component and Factor Loadings}
\begin{itemize}
    \item \textbf{Factor Loadings} indicate how much a variable contributes to a factor. Higher loadings suggest a stronger relationship with the factor.
    \item Example:
\begin{lstlisting}[language=R]
pcUSA$loadings
\end{lstlisting}

\item \textbf{Communalities} measure the proportion of variance a variable shares with all components:
\begin{lstlisting}[language=R]
com1 <- (pcUSA$loadings[1,]*sqrt(eigs))^2
sum(com1[1:2])  # Sum of communalities for the first two components
\end{lstlisting}
\end{itemize}

\section{Factor Analysis of Heptathlon Data}
\begin{itemize}
    \item Factor Analysis is performed on the \texttt{heptathlon} dataset, taking into account different directions for specific variables.
    \item Example:
\begin{lstlisting}[language=R, breaklines=true]
data("heptathlon", package = "HSAUR")
pca_hep <- PCA(heptathlon, quanti.sup = 8)
plot(pca_hep$eig[,1], type = "o", main = "Scree Plot for Heptathlon Data")
\end{lstlisting}
\item \textbf{Practical Tip}:
    \begin{itemize}
        \item Supplementary variables and individuals (\texttt{ind.sup}, \texttt{quanti.sup}) can be included to analyze additional points without affecting the main analysis.
    \end{itemize}
\end{itemize}

\end{document}
