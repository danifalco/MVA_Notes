\documentclass{article}
\usepackage{amsmath}
\usepackage{listings}
\usepackage{xcolor}
\usepackage[a4paper, margin=1in]{geometry}

\definecolor{codebg}{rgb}{0.95,0.95,0.95}
\lstset{backgroundcolor=\color{codebg}, 
        frame=single, 
        basicstyle=\ttfamily,
        keywordstyle=\color{blue},
        commentstyle=\color{gray},
        stringstyle=\color{red},
        showstringspaces=false}

\title{Multidimensional Scaling in R: Comprehensive Guide}
\author{Dani}
\date{}

\begin{document}

\maketitle

\section{Introduction to Multidimensional Scaling (MDS)}
\textbf{Multidimensional Scaling (MDS)} is a set of methods for visualizing the structure of distance data in a lower-dimensional space. The purpose is to project high-dimensional dissimilarities into a 2D or 3D plane while retaining as much information about their relative similarities as possible.

\subsection{Key Types of MDS}
There are two primary types of MDS based on the treatment of dissimilarities:
\begin{itemize}
    \item \textbf{Metric MDS}: Preserves the actual dissimilarity values, making it suitable for data where the numerical distance has a direct interpretation.
    \item \textbf{Non-Metric MDS}: Focuses on preserving the rank order of dissimilarities rather than exact values. This is useful for data where only the order of similarity matters (e.g., preference rankings).
\end{itemize}

\subsection{Choosing Between Metric and Non-Metric MDS}
The decision between Metric and Non-Metric MDS often depends on whether the distance matrix contains meaningful magnitudes. For example:
\begin{itemize}
    \item Use \textbf{Metric MDS} when:
        \begin{itemize}
            \item You have continuous data with interpretable Euclidean distances.
            \item Exact distances matter more than their order.
        \end{itemize}
    \item Use \textbf{Non-Metric MDS} when:
        \begin{itemize}
            \item Dealing with ordinal or categorical data.
            \item Only the relative ranking of distances is meaningful.
            \item You suspect that a non-linear relationship might better capture the patterns in the data.
        \end{itemize}
\end{itemize}

\section{Distance Calculations using the \texttt{cluster} Package}
\subsection{Understanding the \texttt{daisy()} Function}
\begin{itemize}
    \item The \texttt{daisy()} function is highly versatile for computing distance matrices in R, particularly for mixed-type data.
    \item \textbf{Available Distance Metrics}:
    \begin{itemize}
        \item \textbf{Euclidean Distance}: Measures straight-line distance and assumes continuous, numeric data.
        \item \textbf{Gower Distance}: Handles mixed data types (numeric, ordinal, and categorical).
    \end{itemize}
    \item \textbf{Example}:
\begin{lstlisting}[language=R]
library(cluster)
data("iris")

# Calculate Euclidean distance
euclidean_dist <- daisy(iris[, -5], metric = "euclidean")

# Calculate Gower distance
gower_dist <- daisy(iris[, -5], metric = "gower")
\end{lstlisting}
\item \textbf{Practical Consideration}:
    \begin{itemize}
        \item \texttt{daisy()} is often preferable over \texttt{dist()} when working with data frames containing non-numeric variables.
        \item Consider scaling your data when using Euclidean distances to avoid variables with larger scales dominating the result.
    \end{itemize}
\end{itemize}

\section{Principal Coordinates Analysis using \texttt{labdsv}}
\begin{itemize}
    \item \textbf{Principal Coordinates Analysis (PCoA)} is a linear method that represents dissimilarities in Euclidean space.
    \item Ideal for visualizing data with a known distance matrix.
    \item Example using the \texttt{eurodist} dataset:
\begin{lstlisting}[language=R, breaklines=true]
library("labdsv")

# Perform Principal Coordinates Analysis (PCoA)
pco_eu <- pco(eurodist, k = 2)

# Visualization
plot(pco_eu$points[,1], pco_eu$points[,2], type = "n", main = "PCoA of European Cities")
text(pco_eu$points[,1], pco_eu$points[,2], labels(eurodist), cex = 0.8)
\end{lstlisting}
\end{itemize}

\section{Classical MDS using \texttt{cmdscale()} from the \texttt{stats} Package}
\begin{itemize}
    \item Classical MDS (`cmdscale`) is ideal when the goal is to retain the true Euclidean distances.
    \item Computes a configuration that best fits the input dissimilarity matrix.
    \item Example:
\begin{lstlisting}[language=R, breaklines=true]
# Perform Classical MDS on eurodist dataset
mds_eu <- cmdscale(eurodist, eig = TRUE)

# Plotting the MDS solution
plot(mds_eu$points[,1], mds_eu$points[,2], main = "Classical MDS", type = "n")
text(mds_eu$points[,1], mds_eu$points[,2], labels(eurodist), cex = 0.8)
\end{lstlisting}

\item \textbf{Interpretation}:
    \begin{itemize}
        \item The first two components typically capture the majority of the variance.
        \item The eigenvalues indicate the variance explained by each dimension.
        \item Negative eigenvalues suggest poor fit and indicate that the distance matrix might not be truly Euclidean.
    \end{itemize}
\end{itemize}

\section{Non-Metric MDS using \texttt{isoMDS()} from the \texttt{MASS} Package}
\begin{itemize}
    \item Non-Metric MDS (`isoMDS`) preserves the rank order of dissimilarities.
    \item Useful for understanding non-linear relationships in the data.
\begin{lstlisting}[language=R, breaklines=true]
library(MASS)

# Perform Non-Metric MDS
usa_nmds <- isoMDS(dist(USArrests))

# Visualization
plot(usa_nmds$points, main = "Non-Metric MDS on USArrests", type = "n")
text(usa_nmds$points, labels = row.names(USArrests), cex = 0.8)
\end{lstlisting}
\item \textbf{Interpreting Stress Values}:
    \begin{itemize}
        \item The `stress` value indicates how well the 2D configuration matches the original dissimilarities.
        \item Lower stress values (\(< 0.1\)) indicate a good fit, while higher values suggest distortions.
    \end{itemize}
\end{itemize}

\section{Non-Linear Mapping with \texttt{sammon()}}
\begin{itemize}
    \item The \texttt{sammon()} function in R performs a form of non-metric MDS, but it emphasizes preserving local structures.
\begin{lstlisting}[language=R, breaklines=true]
nmds_sam <- sammon(dist(USArrests))
plot(nmds_sam$points, main = "Sammon Mapping on USArrests", type = "n")
text(nmds_sam$points, labels = row.names(USArrests))
\end{lstlisting}
\item \textbf{When to Use Sammon Mapping}:
    \begin{itemize}
        \item When maintaining local neighborhood distances is more important than preserving global structure.
    \end{itemize}
\end{itemize}

\section{3D Visualizations using the \texttt{rgl} Package}
\begin{itemize}
    \item For visualizing MDS results in 3D, use the `rgl` package:
\begin{lstlisting}[language=R, breaklines=true]
library(rgl)

# 3D Visualization
plot3d(mds_eu$points[,1:3], main = "3D MDS Visualization of European Cities")
\end{lstlisting}
\item \textbf{Practical Tip}: Use `rgl` to interactively rotate and explore the 3D configuration.
\end{itemize}

\section{Advanced Considerations: Goodness-of-Fit Analysis}
\begin{itemize}
    \item Use the sum of squared eigenvalues and stress values to evaluate how well the MDS solution represents the original data.
    \item Example:
\begin{lstlisting}[language=R, breaklines=true]
sum(mds_eu$eig[1:2]) / sum(mds_eu$eig)  # Proportion of variance explained by first 2 dimensions
\end{lstlisting}
\item \textbf{Key Insight}: A higher proportion suggests that the first two dimensions capture most of the variance, making the 2D plot a good representation.
\end{itemize}

\end{document}
